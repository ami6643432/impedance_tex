%%%%%%%%%%%%%%%%%%%%%%%%%%%%%%%%%%%%%%%%%%%%%%%%%%%%%%%%%%%%%%%%%%%%%%%%%%%%%%%%
%2345678901234567890123456789012345678901234567890123456789012345678901234567890
%        1         2         3         4         5         6         7         8

\documentclass[letterpaper, 10 pt, conference]{ieeeconf}  % Comment this line out
                                                          % if you need a4paper
%\documentclass[a4paper, 10pt, conference]{ieeeconf}      % Use this line for a4
                                                          % paper

\IEEEoverridecommandlockouts                              % This command is only
                                                          % needed if you want to
                                                          % use the \thanks command
\overrideIEEEmargins
% See the \addtolength command later in the file to balance the column lengths
% on the last page of the document



% The following packages can be found on http:\\www.ctan.org
%\usepackage{graphics} % for pdf, bitmapped graphics files
%\usepackage{epsfig} % for postscript graphics files
% \usepackage{mathptmx} % assumes new font selection scheme installed
%\usepackage{times} % assumes new font selection scheme installed
%\usepackage{amsmath} % assumes amsmath package installed
%\usepackage{amssymb}  % assumes amsmath package installed

%\usepackage{comment}
% \usepackage{commath}
%for \absolute "\abs"

\usepackage{mathtools}
\usepackage{gensymb}
\usepackage[export]{adjustbox}
\usepackage{color}
\usepackage{graphics} % for pdf, bitmapped graphics files
\usepackage{epsfig} % for postscript graphics files
% \usepackage{mathptmx} % assumes new font selection scheme installed
\usepackage{times} % assumes new font selection scheme installed
\usepackage{amsmath} % assumes amsmath package installed
\usepackage{amssymb}  % assumes amsmath package installed
\usepackage{leftidx}
\newtheorem{theorem}{Theorem}
\newtheorem{remark}{Remark}
\newtheorem{assum}{Assumption}
\usepackage{amsmath}
\newtheorem{property}{Property}
\newtheorem{observation}{Observation}
\newtheorem{mydef}{Definition}
% \usepackage[linesnumbered,ruled]{algorithm2e}
\usepackage{siunitx}
\usepackage{cite}[noadjust]
\usepackage{nomencl}
\usepackage{siunitx}
\makenomenclature

\DeclareMathOperator{\diag}{diag}
\newcommand{\RomanNumeralCaps}[1]
    {\MakeUppercase{\romannumeral #1}}

\usepackage{algorithm,algorithmic}
\usepackage{tablefootnote}

\title{\LARGE \bf
Impedance Control of Euler-Lagrange Systems with Unknown Dynamics
}



\author{---------------% <-this % stops a space
% \thanks{*This work was not supported by any organization}% <-this % stops a space
 \thanks{ 
       {\tt\small }}
}


\begin{document}



\maketitle
\thispagestyle{empty}
\pagestyle{empty}
\setlength{\belowcaptionskip}{-10pt} 


%%%%%%%%%%%%%%%%%%%%%%%%%%%%%%%%%%%%%%%%%%%%%%%%%%%%%%%%%%%%%%%%%%%%%%%%%%%%%%%%
\begin{abstract}
%In this paper a distributed adaptive control framework is proposed for the aerial manipulator system for pick and place task. The controller uses the coupled dynamic of UAV-Manipulator which is highly non-linear and complex. The proposed design did not require any a priori knowledge of structure  and bound of uncertainty.

\end{abstract}
%%%%%%%%%%%%%%%%%%%%%%%%%%%%%%%%%%%%%%%%%%%%%%%%%%%%%%%%%%%%%%%%%%%%%%%%%%%%%%%%
\section{INTRODUCTION}

.
.
.
.

% The rest of the paper is organised as follows: Sect. \RomanNumeralCaps{2} 
% describes the quadrotor dynamics and the control problem;
% Sects. \RomanNumeralCaps{3}  and \RomanNumeralCaps{4} detail the proposed control framework and
% its stability analysis respectively; Sect. \RomanNumeralCaps{5} provides comparative simulation results while Sect. \RomanNumeralCaps{6} provides concluding
% remarks.

% The following notations are used in this paper: $\lambda_{min}(\cdot)$ and $||\cdot||$ denote he minimum eigenvalue and Euclidean norm, and pseudo inverse of of $(\cdot)$, respectively. $I$ denotes identity matrix with appropriate dimension.

\section{DESIRED DYNAMICS AND PROBLEM FORMULATION}


In various robot-environment interaction scenarios we desire the ability to have adaptable interactions with environments that are not completely known or predictable.
To achieve this simultaneous force-motion control in task space we often desire to control the way the robot interacts with the environment, or essentially we try to control the dynamics of the system.

The simple second order dynamics, similar to the standard spring-mass-damper, are well studied and understood, and hence a very popular choice of dynamics to be imposed on the system.

\subsection{Conventional UAM Model and Dynamics}

We have considered consider an generalized aerial-manipulator system with a $n$ degrees-of-freedom (DoFs) manipulator system as in Fig. \ref{robot_model}. For this system, let the position $p$ and orientation $q$ (in Euler angles) of the quadrotor be defined as $p \triangleq \begin{bmatrix}
	x & y & z
	\end{bmatrix}^T \in \mathbb{R}^{3}$ and  ${q} \triangleq \begin{bmatrix}
	    \phi & \theta & \psi
	\end{bmatrix}^T \in \mathbb{R}^{3}$ and we define ${\alpha} \triangleq \begin{bmatrix}
	    \alpha_1 & \alpha_2 & .. & \alpha_n
	\end{bmatrix}^T \in \mathbb{R}^{n}$ for the $n$-link manipulator system. The Euler-Lagrangian dynamical model for this system can be represented as \cite{arleo2013control}
 
\begin{equation} \label{General_EL_dynamics}
M(\chi(t))\ddot{\chi}(t) + C(\chi(t), \dot{\chi}(t))\dot{\chi}(t) + g(\chi(t)) + d(t) = \tau + \tau_ {ext}   
\end{equation}
where $\chi = \begin{bmatrix} p^T & q^T & \alpha^T\end{bmatrix}^T \in \mathbb{R}^{6+n}$ is the generalize state vector; $M\in \mathbb{R}^{(6+n)\times (6+n)}$ represents the
inertia matrix; $C\in \mathbb{R}^{(6+n)\times (6+n)}$ is the coriolis and centrifugal matrix; $g\in \mathbb{R}^{6+n}$ is the
vector of gravity forces; $d\in \mathbb{R}^{6+n}$ is the bounded external disturbance; $\tau\in \mathbb{R}^{6+n}$ is the control input vector and $\tau_ {ext} \in \mathbb{R}^{6+n}$ is the torque due to external forces acting on the end-effector. These dynamics terms can be split as:
\begin{subequations}\label{split_2}
\begin{align}
&M = \begin{bmatrix}
    M_{pp} & M_{pq} & M_{p\alpha} \\
    M_{pq}^T & M_{qq} & M_{q\alpha} \\
    M_{p\alpha}^T & M_{q\alpha}^T & M_{\alpha \alpha}
    \end{bmatrix},~\begin{matrix}
M_{pp}, M_{qq}, M_{pq} \in \mathbb{R}^{3\times3}\\ 
M_{p\alpha},M_{q\alpha} \in \mathbb{R}^{3 \times n}\\ 
M_{\alpha \alpha} \in \mathbb{R}^{n \times n}
\end{matrix}\\
%\end{equation}
%with $M_{pp}, M_{qq}, M_{pq} \in \mathbb{R}^{3\times3}$, $M_{p\alpha},M_{q\alpha} \in \mathbb{R}^{3\times2}$ and $M_{\alpha \alpha} \in \mathbb{R}^{2\times2}$ and % Similarly, matrix C and vector g in (\ref{EL_dynamics}) can be expressed as
&C = \begin{bmatrix}
    C_p \\
    C_q \\
    C_\alpha
    \end{bmatrix}, ~\begin{matrix} C_p , C_q \in \mathbb{R}^{3 \times (6+n)} \\C_\alpha \in \mathbb{R}^{n \times (6+n)}  \end{matrix}\\
&g = \begin{bmatrix}
    g_p \\
    g_q \\
    g_\alpha
    \end{bmatrix},
    d = \begin{bmatrix}
    d_p \\
    d_q \\
    d_\alpha
    \end{bmatrix}, ~\begin{matrix} g_p, g_q,d_p,d_q \in \mathbb{R}^{3}\\ g_\alpha, d_\alpha \in \mathbb{R}^{n}\end{matrix}\\
& \tau = \begin{bmatrix}
    \tau_p \\
    \tau_q \\
    \tau_\alpha
    \end{bmatrix}, ~\begin{matrix} \tau_p, \tau_q \in \mathbb{R}^{3}\\ \tau_\alpha \in \mathbb{R}^{n}\end{matrix}.\\
& \tau_{ext} = \begin{bmatrix}
    0 \\
    0 \\
    J_{\alpha}^T F_{ext}\\
    \end{bmatrix}, ~\begin{matrix} F_{ext} \in \mathbb{R}^{3}\\ J_\alpha \in \mathbb{R}^{n \times 3}\end{matrix}.
\end{align}
\end{subequations}
%with and $g_p, g_q,d_p,d_q, \tau_p, \tau_q \in \mathbb{R}^{3}$, $g_\alpha, d_\alpha, \tau_\alpha \in \mathbb{R}^{2}$.
%The gravity vector $g(\chi)$ and disturbance vector $d$ is defined as $g(\chi) \triangleq \begin{bmatrix}
%	    g^T_p & g^T_q & g^T_\alpha
%	\end{bmatrix}^T \in \mathbb{R}^{8}$ and $d \triangleq \begin{bmatrix}
%	    d^T_p & d^T_q & d^T_\alpha
%	\end{bmatrix}^T \in \mathbb{R}^{8}$ where ${g_p} \in\mathbb{R}^{3}$, ${g_q} \in\mathbb{R}^{3}$, ${g_\alpha} \in\mathbb{R}^{2}$ and ${d_p} \in\mathbb{R}^{3}$, ${d_q} \in\mathbb{R}^{3}$, ${d_\alpha} \in\mathbb{R}^{2}$.
%The control input is represented as $\tau \triangleq \begin{bmatrix} \tau_p^T & \tau_q^T & \tau^T_\alpha \end{bmatrix}^T \in \mathbb{R}^{8}$ where 
Where $J_{\alpha}$ is the analytic jacobian for the manipulator.

Here,    $\tau_{\alpha} \triangleq
	\begin{bmatrix}
	\tau_{\alpha_1} & \tau_{\alpha_2}  & .. & \tau_{\alpha_n}
	\end{bmatrix}^T$ is used for the control input for the manipulator; $\tau_q  \triangleq 
	\begin{bmatrix}
	u_2(t) & u_3(t) & u_4(t)
	\end{bmatrix}^T$ is used as the control inputs for roll, pitch and yaw of the quadrotor; ${\tau_{p}} = {R^W_B U}$ is the generalized control input for quadrotor position in Earth-fixed frame, such that ${U}(t)\triangleq
	\begin{bmatrix}
	0 & 0 & u_1(t)
	\end{bmatrix}^T\in \mathbb{R}^3$ being the force vector in body-fixed frame and ${R_B^W} \in\mathbb{R}^{3\times3}$ being the $Z-Y-X$ Euler angle rotation matrix describing the rotation from the body-fixed coordinate frame to the Earth-fixed frame, given by
	\begin{align}
	{R_B^W} =
	\begin{bmatrix}
	c_{\psi}c_{\theta} & c_{\psi}s_{\theta}s_{\phi} - s_{\psi}c_{\phi} & c_{\psi}s_{\theta}c_{\phi} + s_{\psi}s_{\phi} \\
	s_{\psi}c_{\theta} & s_{\psi}s_{\theta}s_{\phi} + c_{\psi}c_{\phi} & s_{\psi}s_{\theta}c_{\phi} - c_{\psi}s_{\phi} \\
	-s_{\theta} & s_{\phi}c_{\theta} & c_{\theta}c_{\phi}
	\end{bmatrix}, \label{rot_matrix}
	\end{align}
	where $c_{(\cdot)} , s_{(\cdot)}$ and denote $\cos{(\cdot)} , \sin{(\cdot)}$ respectively.

 \subsection{Distributed UAM Dynamics}

For ease of control design and analysis and using \eqref{split_2}, the UAM dynamics \eqref{General_EL_dynamics} can be re-written as
\begin{subequations} \label{dynamics}
	\begin{align}
	M_{pp}\ddot{p} &+ M_{pq}\ddot{q} + M_{p\alpha}\ddot{\alpha} + C_{p}\dot{\chi} + g_{p} + d_{p} &= \tau_{p} \label{pos}\\
	M_{pq}^T\ddot{p} &+ M_{qq}\ddot{q} + M_{q\alpha}\ddot{\alpha} + C_{q}\dot{\chi} + g_{q} + d_{q} &= \tau_{q} \label{att} \\
	M_{p\alpha}^T\ddot{p} &+ M_{q\alpha}^T\ddot{q} + M_{\alpha\alpha}\ddot{\alpha} + C_{\alpha} \dot{\chi} + g_{\alpha} + d_{\alpha} &= \tau_{\alpha} + \tau_{\alpha ext} \label{man}\\
	\tau_{p} &= R^W_B U \label{conv}\\
	\tau_{\alpha ext} &= J_{\alpha}^T F_{ext}
	\end{align}
\end{subequations}


\textcolor{red}{Need to format the equation numbers into this.}

where (\ref{pos}), (\ref{att}) and (\ref{man}) represent the quadrotor position dynamics, quadrotor attitude dynamics and manipulator dynamics along with their interactions, respectively. 

The following standard system properties hold via Euler-Lagrange mechanics \cite{spong2008robot}:
\begin{property} \label{prop_1}
The matrix {$M_(\chi)$} is uniformly positive definite and $ \exists \underline{m},\overline{m} \in \mathbb{R}^{+}$ such that $0 < \underline{m} I \leq M(\chi) \leq \overline{m} I$. 
\end{property}

\begin{property} \label{prop_3}
$\exists \bar{c}, \bar{g}, \bar{d} \in\mathbb{R}^{+}$ such that $||C (\chi)|| \leq \bar{c}||\dot{\chi}||$, $||g (\chi)|| \leq \bar{g}$ and $||d(t)|| \leq \bar{d}$. % where $\chi$ is defined in eq (\ref{EL_dynamics}).
\end{property}

The following assumption highlights the available knowledge of various system parameters for the control design:

\begin{assum}[Uncertainty] \label{assum_1}
The system dynamics terms $M, C, g, d$
and their bounds $\overline{m}, \underline{m}, \bar{c}, \bar{g}, \bar{d}$ defined in Properties 1-2 are unknown for control design.
\end{assum}

\begin{assum} \label{assum_2}
The desired trajectories $\chi_d = \begin{bmatrix} p_d^T & q_d^T & \alpha_d^T\end{bmatrix}^T$ and their time-derivatives $\dot{\chi}_d, \ddot{\chi}_d$ are designed to
be sufficiently bounded. Furthermore, $\chi, \dot{\chi}, \ddot{\chi}$ are available for feedback.
\end{assum}


\textit{Control Problem} Under Properties 1-2 and Assumptions 1-2, to design a distributed adaptive control framework for the aerial manipulator system (\ref{dynamics}), \textcolor{red}{with an important objective being to impose desired impedance dynamics on the manipulator for the attenuation of forces arising from environmental interactions.}

The following section solves the control problem.


% \subsection{System Dynamics}

% Considering a manipulator with $n$ degrees-of-freedom (DoFs). For this system, let the position in cartesian space $r$ and the state in joint space $q$ be defined as $r \triangleq \begin{bmatrix}
% 	x & y & z
% 	\end{bmatrix}^T \in \mathbb{R}^{3}$ and  ${q} \triangleq  \begin{bmatrix}
% 	    {q}_1 & {q}_2 & .. & {q}_n
% 	\end{bmatrix}^T \in \mathbb{R}^{n}$ for the $n$-link manipulator system. The Euler-Lagrangian dynamical model for this system in task space can be represented as 
%  % \cite{arleo2013control}
 
% \begin{equation} \label{EL_dynamics}
% M(\chi(t))\ddot{\chi}(t) + C(\chi(t), \dot{\chi}(t))\dot{\chi}(t) + g(\chi(t)) + d(t) = \tau + \tau_ {ext}   
% \end{equation}

% $M$ represents the inertia matrix; $C$ is the coriolis and centrifugal matrix; $g$ is the vector of gravity forces; $F_ext$ is the external force measured on the tool tip and $F_c =\quad (J^T)^{-1} \tau$ is the control force. 

\section{Controller Design and Analysis}

The proposed control framework consists of designing (i)  quadrotor position control, (ii) quadrotor attitude control and (iii) the manipulator control as per the dynamics (\ref{dynamics}). Note that such design process is simultaneous in nature and not decoupled.  
%A task for the aerial-manipulator system modeled in the previous section is usually specified in terms of a desired trajectory $\chi_d = \begin{bmatrix} p_d^T & q_d^T & \theta_d^T\end{bmatrix}^T \in \mathbb{R}^{8}$ where $p_d \in \mathbb{R}^3$, $q_d \in \mathbb{R}^3$ and $\alpha_d \in \mathbb{R}^2$
Defining tracking error as  
\begin{align} 
  e \triangleq \chi(t) - \chi_d(t),~\xi(t) \triangleq \begin{bmatrix}
	e(t) & \dot{e}(t)
	\end{bmatrix},
\end{align}  \label{err}
we elaborate the proposed control designs in the following subsections.
	
\subsection{Quadrotor Position Control}
% For control design purpose, the quadrotor position sub-dynamics (\ref{dynamics}a) is rearranged as
% \begin{equation} \label{dynamics_p}
% \bar{M}_{pp}\ddot{p} - E_{p} = \tau_{p}    
% \end{equation}
% where $\bar{M}_{pp}$ is a user-defined constant positive definite matrix and $E_{p} \triangleq -\{(M_{pp} - \bar{M}_{pp})\ddot{p} + M_{pq}\ddot{q} + M_{p\alpha}\ddot{\alpha} + C_{p}\dot{\chi} + g_{p} + d_{p} \}$. The selection of $\bar{M}_{pp}$ would be discussed later (cf. Remark \ref{mass}).
% \textcolor{red}{intentionally making $E$ in negative for lyapunov proof}

Taking the quadrotor position tracking error as $e_p(t) \triangleq p(t) - p_d(t)$, let us define an error variable as
\begin{align}
     s_p &= \dot{e}_p + \Phi_p e_p \label{eq:s_p}
 \end{align}
where $\Phi_{p} \in \mathbb{R}^{3 \times 3}$ is a positive definite gain matrix. Multiplying the time derivative of (\ref{eq:s_p}) by $\bar{M}_{pp}$ and  using (\ref{eq:tau_p}) yields
\begin{align}
\bar{M}_{pp} \dot{s}_p &= \bar{M}_{pp}(\ddot{p}- \ddot{p}_d+ \Phi_{p} {\dot{e}_p}) \nonumber \\ 
&= {\tau_{p}} + {E}_{p} - \bar{M}_{pp}(\ddot{p}_d- \Phi_{p} {\dot{e}_p}), \label{eq:sp_dot}
\end{align}
where $\bar{M}_{pp}$ is a user-defined constant positive definite matrix and $E_{p} \triangleq -\{(M_{pp} - \bar{M}_{pp})\ddot{p} + M_{pq}\ddot{q} + M_{p\alpha}\ddot{\alpha} + C_{p}\dot{\chi} + g_{p} + d_{p} \}$. The selection of $\bar{M}_{pp}$ would be discussed later (cf. Remark \ref{mass}).




% Let us define the position tracking error as $e_p(t) \triangleq p(t) - p_d(t)$ and  
% an error variable $r_{p}$ as
% \begin{equation} \label{err_variable_p}
% r_{p} \triangleq B^T_p P_{p} \xi_p    
% \end{equation}
% where $\xi_p(t) \triangleq \begin{bmatrix}
% 	e^T_p(t) & \dot{e}^T_p(t)
% 	\end{bmatrix}^T$, 
% 	$B_p \triangleq \begin{bmatrix}
% 	0 & I
% 	\end{bmatrix}^T$; $P_{p} > 0$ is the solution to the Lyapunov equation $A^T_{p} P_{p} + P_{p} A_{p} = -Q_{p} $ for some $Q_{p} > 0$ with $A_{p} \triangleq \begin{bmatrix}
% 	0 & I \\ -\lambda_{p1} & -\lambda_{p2} 
% 	\end{bmatrix}$. Here, $\lambda_{p1}$ and $\lambda_{p2}$ are two user-defined positive definite gain matrices and their positive definiteness guarantees that $A_{p}$ is Hurwitz.

The control law is proposed as
\begin{subequations}\label{ct1}
\begin{align}
\tau_{p} &= -\Lambda_{p} s_p - \Delta \tau_{p} + \bar{M}_{pp}(\ddot{p}_d- \Phi_{p} {\dot{e}_p}),  \label{tau_p}\\
\Delta \tau_{p} &= \begin{cases}
    \rho_{p} \frac{s_{p}}{||s_{p}||}       & ~ \text{if } || s_{p}|| \geq \varpi_p\\
    \label{del_p}
    \rho_{p} \frac{s_{p}}{\varpi_p}       & ~ \text{if } || s_{p}|| < \varpi_p\\
    \end{cases}
\end{align}
\end{subequations}
where $\Lambda_{p}$ is a user-defined positive definite gain matrix and $\varpi_{p} > 0$ is a scalar used to avoid chattering; $\rho_{p}$ tackles system uncertainties, whose design will be discussed later.


% Substituting (\ref{tau_p}) into (\ref{dynamics_p}) yields
% \begin{equation}
% \ddot{e}_p = -\Lambda_{p} \xi_p - \Delta \tau_{p} + \Upsilon_{p} \label{err1}  
% \end{equation}
% where $\Upsilon_{p} \triangleq - \bar{M}_{pp}^{-1} E_p$ is defined as the overall uncertainty. 

% Using system Properties 1-2 and Assumption \ref{assum_2} one can verify
%  \begin{align}
% ||\Upsilon_{p}|| &\leq ||\bar{M}_{pp}^{-1}|| (||(M_{pp} - \bar{M}_{pp} )|| ||\ddot{p}|| + || M_{pq}|| ||\ddot{q}|| \nonumber \\ &+ || M_{p\alpha}|| ||\ddot{\alpha}|| + ||C_p|| ||\dot{\chi}|| + ||{g}_p|| + ||{d}_p|| ). \label{cp}
%  \end{align}
 
Using Property 2 and the relation (\ref{split_2}) we have $||C_{p}|| \leq ||C|| \leq \bar{c}||\dot{\chi}||$, $||g_{p}|| \leq ||g|| \leq \bar{g}$ and $||d_{p}|| \leq ||d|| \leq \bar{d}$. Using these relations, the inequalities $||\ddot{\chi}|| \geq ||\ddot{p}||$, $||\ddot{\chi}|| \geq ||\ddot{q}||$ and $||\ddot{\chi}|| \geq ||\ddot{\alpha}||$, $||\xi|| \geq ||\dot{e}||$, $||\xi|| \geq ||{e}||$ and substituting $\dot{\chi} = \dot{e} + \dot{\chi}_d$ into (\ref{cp}) yields
\begin{align} 
||E_{p}|| &\leq K_{p0}^*  +K_{p1}^*||\xi||+ K_{p2}^*||\xi||^2 + K_{p3}^*||\ddot{\chi}||\label{up_bound_p} 
%&\triangleq \mathbf{Y}^T \mathbf{K_{p}^*}
\end{align}
%where $\mathbf{Y} = \begin{bmatrix}
%1 & ||\xi|| & ||\xi||^2 & ||\ddot{\chi}||
%\end{bmatrix}^T$ and $\mathbf{K_{p}^*} = \begin{bmatrix}
%K_{p0}^* & K_{p1}^* & K_{p2}^* & K_{p3}^*
%\end{bmatrix}$. The scalars $K_{pi}^* \in \mathbb{R}^+, ~i =0,1,2,3$ are unknown and they are defined as
\begin{align*}
 K_{p0}^* &= \bar{g} + \bar{d} + \bar{c}||\dot{\chi}_d||^2,\\
K_{p1}^* &= 2\bar{c}|| \dot{\chi}_d||,~
K_{p2}^* = \bar{c}, \\
K_{p3}^* &=  (||(M_{pp} - \bar{M}_{pp} )||+ ||M_{pq}|| + ||M_{p\alpha}||)
\end{align*}
are unknown scalars. Based on the upper bound structure in (\ref{up_bound_p}), the gain $\rho_{p}$ in (\ref{del_p}) is designed as
\begin{align}
\rho_{p} = \hat{K}_{p0} + \hat{K}_{p1}||\xi|| + \hat{K}_{p2}||\xi||^2 + \hat{K}_{p3}||\ddot{\chi}|| + \zeta_{p} + \gamma_{p} \label{rho_p} 
\end{align}
%\nonumber \\
%&\triangleq \mathbf{Y^T \hat{K}_{p}} + \zeta_p 

where $\hat{K}_{pi}$ are the estimates of $K_{pi}^*$ $i=0,1,2,3$, and $\zeta_{p}$, $\gamma_{p}$ are auxiliary gain used for closed-loop stabilization (cf. Remark \ref{rem_zeta}). The gains $\hat{K}_{pi}$ are adapted via the following laws:
\begin{subequations}\label{adaptive_law_p}
\begin{align}
&\dot{\hat{K}}_{pia} = ||s_{p}||||\xi||^i - \nu_{pia} \hat{K}_{pia}, \dot{\hat{K}}_{pi\bar{a}} = 0,  i = 0,1,2
 \\
&\dot{\hat{K}}_{p3a} =  ||s_{p}||||\ddot{\chi}|| - \nu_{p3a} \hat{K}_{p3a}, \dot{\hat{K}}_{p3\bar{a}} = 0\\
&\dot{\zeta}_{pa} = -(1 + \hat{K}_{p3a}||s_{a}||||\ddot{\chi}||)\zeta_{pa} + \bar{\epsilon}_{pa}, \dot{\zeta}_{p\bar{a}} = 0    \\ \label{zeta1}
& \dot{\gamma}_{pa} = 0,  \dot{\gamma}_{p\bar{a}} = \left(1+ \frac{\varrho_{p\bar{a}}}{2}\sum \limits_{i=0}^{3} {\hat{K}_{pi\bar{a}}}^2 \right) \gamma_{p\bar{a}} +\epsilon_{p\bar{a}} \\ 
& \text{with}~ \nu_{pi} > \varrho_{p}/2, i = 0,1,2\\
&~\hat{K}_{pi} (t_0) > 0, ~\zeta_{p} (t_0)  = \bar{\zeta}_{p} > \bar{\epsilon}_{p}, \gamma_{p} (t_0) = \bar{\gamma}_{p} > \epsilon_{p}
\end{align}
\end{subequations}
where a and $\bar{a} \in \Omega \backslash {a}$ denote the active and inactive subsystems respectively; $\nu_{pia},\bar{\epsilon}_{pa}, \epsilon_{p\bar{a}}  \in\mathbb{R}^{+}$ are user-defined scalars and $\varrho_{p} \triangleq \frac{\min \lbrace \lambda_{\min}(  \Lambda_{{p}} ), \lambda_{\min}( \Phi_{{p}} ) \rbrace}{\max \lbrace \bar{M}_{pp}, 1 \rbrace}$ $\forall  \in \Omega$ and $t_0$ is the initial time and from \ref{} and intial condition \ref{}, it can be verified that $\exists \underline{\zeta}_{p}, \underline{\gamma}_{p} \in \mathbb{R}^{+}$ such that
\begin{subequations}
\begin{align}
&\hat{K}_{pi}(t) \geq 0, ~~ 0 < \underline{\zeta}_{p} \leq \zeta_{p} (t) <  \bar{\zeta}_{p}, \nonumber \\
\text{and}~~ & 0 < \underline{\gamma}_{p} \leq \gamma_{p} (t) <  \bar{\gamma}_{p} ~~ \forall t \geq t_0.
\end{align}
\end{subequations}


% \textcolor{red}{Doubt: Role of $\gamma$ and $\varrho_{p}$ }


\subsection{Quadrotor Attitude Control}
%To design the attitude controller, the desired roll ($\phi_d$) and pitch ($\theta_d$) angles are to be generated first. This process involves defining an intermediate coordinate frame as the first step (cf. \cite{mellinger2011minimum}):
%\begin{subequations}\label{int_co}
%\begin{align}
%    z_B &= \frac{\tau_p}{||\tau_p||},~
 %   y_A = \begin{bmatrix}
 %   -s_{\psi_d} & c_{\psi_d} & 0
%\end{bmatrix}^T \\
%    x_B &= \frac{y_A \times z_B}{||y_A \times z_B||} ,~
%    y_B = z_B \times x_B
%\end{align}
%\end{subequations}
%where $y_A$ is the $y$-axis of the intermediate coordinate frame and ($x_B, y_B,z_B$) are the ($x,y,z$)-axis of the body-fixed coordinate frame. Given the desired yaw angle $\psi_d (t)$ and using (\ref{int_co}), $\phi_d (t)$ and $\theta_d (t)$ can be determined using the desired body frame axes as described in \cite{mellinger2011minimum}. % (the details are omitted here due to lack of space). 

To achieve the attitude control, the tracking error in orientation/attitude is defined as \cite{mellinger2011minimum}
\begin{align}
    e_q &= {((R_d)^T R_B^W - (R_B^W)^T R_d)}^{v}  \\
    \dot{e}_q & = \dot{q} - R_d^T R_B^W \dot{q}_d
\end{align}
where $(.)^v$ represents \textit{vee} map, which converts elements of $SO(3)$ to $\in{\mathbb{R}^3}$ and $R_d$ is the rotation matrix as in (\ref{rot_matrix}) evaluated at ($\phi_d, \theta_d, \psi_d$).

The quadrotor attitude sub-dynamics (\ref{dynamics}b) is rearranged as
\begin{equation} \label{dynamics_q}
\bar{M}_{qq}\ddot{q} + E_{q} = \tau_{q}    
\end{equation}
where $\bar{M}_{qq}$ is a user-defined constant positive definite matrix (cf. Remark \ref{mass}) and $E_{q} \triangleq (M_{qq} - \bar{M}_{qq})\ddot{q} + M_{pq}^T\ddot{p} + M_{q\alpha}\ddot{\alpha} + C_{q}\dot{\chi} + g_{q} + d_{q}$. %The selection of $\bar{M}_{qq}$ would be discussed later.

For quadrotor attitude tracking error, let us define an error variable as
\begin{align}
     s_q &= \dot{e}_q + \Phi_q e_q \label{eq:s_p}
 \end{align}
where $\Phi_{q} \in \mathbb{R}^{3 \times 3}$ is a positive definite gain matrix. Multiplying the time derivative of (\ref{eq:s_q}) by $\bar{M}_{qq}$ and  using (\ref{eq:tau_q}) yields
\begin{align}
\bar{M}_{qq} \dot{s}_q &= \bar{M}_{qq}(\ddot{q}- \ddot{q}_d+ \Phi_{q} {\dot{e}_q}) \nonumber \\ 
&= {\tau_{q}}- {E}_{q} - \bar{M}_{qq}(\ddot{q}_d- \Phi_{q} {\dot{e}_q}), \label{eq:sq_dot}
\end{align}

The control law is proposed as
\begin{subequations}\label{ct2}
\begin{align}
\tau_{q} &= -\Lambda_{q} s_q + \Delta \tau_{q} + \bar{M}_{qq}(\ddot{q}_d- \Phi_{q} {\dot{e}_q}),  \label{tau_q}\\
\Delta \tau_{q} &= \begin{cases}
    \rho_{q} \frac{s_{q}}{||s_{q}||}       & ~ \text{if } || s_{q}|| \geq \varpi_q\\
    \label{del_q}
    \rho_{q} \frac{s_{q}}{\varpi_q}       & ~ \text{if } || s_{q}|| < \varpi_q\\
    \end{cases}
\end{align}
\end{subequations}
where $\Lambda_{q}$ is a user-defined positive definite gain matrix and $\varpi_{q} > 0$ is a scalar used to avoid chattering; $\rho_{q}$ tackles system uncertainties, whose design will be discussed later.
Using Property 2 and the relation (\ref{split_2}) we have $||C_{q}|| \leq ||C|| \leq \bar{c}||\dot{\chi}||$, $||g_{q}|| \leq ||g|| \leq \bar{g}$ and $||d_{q}|| \leq ||d|| \leq \bar{d}$. Using these relations, the inequalities $||\ddot{\chi}|| \geq ||\ddot{p}||$, $||\ddot{\chi}|| \geq ||\ddot{q}||$ and $||\ddot{\chi}|| \geq ||\ddot{\alpha}||$, $||\xi|| \geq ||\dot{e}||$, $||\xi|| \geq ||{e}||$ and substituting $\dot{\chi} = \dot{e} + \dot{\chi}_d$ into (\ref{cp}) yields
\begin{align} 
||E_{q}|| &\leq K_{q0}^*  +K_{q1}^*||\xi||+ K_{q2}^*||\xi||^2 + K_{q3}^*||\ddot{\chi}||\label{up_bound_q} 
%&\triangleq \mathbf{Y}^T \mathbf{K_{p}^*}
\end{align}
where
\begin{align*}
 K_{q0}^* &= \bar{g} + \bar{d} + \bar{c}||\dot{\chi}_d||^2,\\
K_{q1}^* &= 2\bar{c}|| \dot{\chi}_d||,~
K_{q2}^* = \bar{c}, \\
K_{q3}^* &=  (||(M_{qq} - \bar{M}_{qq} )||+ ||M^T_{pq}|| + ||M_{q\alpha}||)
\end{align*}
\textcolor{red}{$K_{p0}^* = K_{q0}^*$}\\
are unknown scalars. Based on the upper bound structure in (\ref{up_bound_q}), the gain $\rho_{q}$ in (\ref{del_q}) is designed as
\begin{align}
\rho_{q} = \hat{K}_{q0} + \hat{K}_{q1}||\xi|| + \hat{K}_{q2}||\xi||^2 + \hat{K}_{q3}||\ddot{\chi}|| + \zeta_{q} + \gamma_{q} \label{rho_q} 
\end{align}


where $\hat{K}_{qi}$ are the estimates of $K_{qi}^*$ $i=0,1,2,3$, and $\zeta_{q}$, $\gamma_{q}$ are auxiliary gain used for closed-loop stabilization (cf. Remark \ref{rem_zeta}). The gains $\hat{K}_{qi}$ are adapted via the following laws:
\begin{subequations}\label{adaptive_law_q}
\begin{align}
&\dot{\hat{K}}_{qia} = ||s_{q}||||\xi||^i - \nu_{qia} \hat{K}_{qia}, \dot{\hat{K}}_{qi\bar{a}} = 0,  i = 0,1,2
 \\
&\dot{\hat{K}}_{q3a} =  ||s_{q}||||\ddot{\chi}|| - \nu_{q3a} \hat{K}_{q3a}, \dot{\hat{K}}_{q3\bar{a}} = 0\\
&\dot{\zeta}_{qa} = -(1 + \hat{K}_{q3a}||s_{q}||||\ddot{\chi}||)\zeta_{qa} + \bar{\epsilon}_{qa}, \dot{\zeta}_{q\bar{a}} = 0    \\ \label{zeta1}
& \dot{\gamma}_{qa} = 0,  \dot{\gamma}_{q\bar{a}} = \left(1+ \frac{\varrho_{q\bar{a}}}{2}\sum \limits_{i=0}^{3} {\hat{K}_{qi\bar{a}}}^2 \right) \gamma_{q\bar{a}} +\epsilon_{q\bar{a}} \\ 
& \text{with}~ \nu_{qi} > \varrho_{q}/2, i = 0,1,2\\
&~\hat{K}_{qi} (t_0) > 0, ~\zeta_{q} (t_0)  = \bar{\zeta}_{q} > \bar{\epsilon}_{q}, \gamma_{q} (t_0) = \bar{\gamma}_{q} > \epsilon_{q}
\end{align}
\end{subequations}
where a and $\bar{a} \in \Omega \backslash {a}$ denote the active and inactive subsystems respectively; $\nu_{qia},\bar{\epsilon}_{qa}, \epsilon_{q\bar{a}}  \in\mathbb{R}^{+}$ are user-defined scalars and $\varrho_{q} \triangleq \frac{\min \lbrace \lambda_{\min}( \Lambda_{{q}} ), \lambda_{\min}( \Phi_{{q}} ) \rbrace}{\max \lbrace \bar{M}_{qq}, 1 \rbrace}$ $\forall  \in \Omega$ and $t_0$ is the initial time and from \ref{} and intial condition \ref{}, it can be verified that $\exists \underline{\zeta}_{q}, \underline{\gamma}_{q} \in \mathbb{R}^{+}$ such that
\begin{subequations}
\begin{align}
&\hat{K}_{qi}(t) \geq 0, ~~ 0 < \underline{\zeta}_{q} \leq \zeta_{q} (t) <  \bar{\zeta}_{q}, \nonumber \\
\text{and}~~ & 0 < \underline{\gamma}_{q} \leq \gamma_{q} (t) <  \bar{\gamma}_{q} ~~ \forall t \geq t_0.
\end{align}
\end{subequations}


\subsection{Manipulator Impedance Control}
For control design purpose, the UAM manipulator sub-dynamics (\ref{dynamics}c) is rearranged as
\begin{equation} \label{dynamics_q}
\bar{M}_{\alpha\alpha}\ddot{\alpha} + E_{\alpha} = \tau_{\alpha}   + \tau_{\alpha{ext}}    
\end{equation}
where $\bar{M}_{\alpha\alpha}$ is a user-defined constant positive definite matrix and $E_\alpha \triangleq (M_{\alpha\alpha} - \bar{M}_{\alpha\alpha})\ddot{\alpha} + M_{p\alpha}^T\ddot{p} + M_{q\alpha}^T\ddot{q} + C_\alpha\dot{\chi} + g_\alpha + d_\alpha $. The selection of $\bar{M}_{pp}$ would be discussed later (cf. Remark \ref{mass}).

Taking the UAM manipulator tracking error as $e_\alpha(t) \triangleq p(t) - p_\alpha(t)$, let us define an error variable as
\begin{align}
     s_\alpha &= \dot{e}_\alpha + \Phi_\alpha e_\alpha \label{eq:s_alpha}
 \end{align}
where $\Phi_{\alpha} \in \mathbb{R}^{3 \times 3}$ is a positive definite gain matrix. Multiplying the time derivative of (\ref{eq:s_alpha}) by $\bar{M}_{\alpha\alpha}$ and  using (\ref{eq:tau_alpha}) yields
\begin{align}
\bar{M}_{\alpha\alpha} \dot{s}_\alpha &= \bar{M}_{\alpha\alpha}(\ddot{\alpha}- \ddot{\alpha}_d+ \Phi_{\alpha} {\dot{e}_\alpha}) \nonumber \\ 
&= {\tau_{\alpha}}- {E}_{\alpha} - \bar{M}_{\alpha\alpha}(\ddot{\alpha}_d- \Phi_{\alpha} {\dot{e}_\alpha}), \label{eq:salpha_dot}
\end{align}


The control law is proposed as
\begin{subequations}\label{ct3}
\begin{align}
\tau_{\alpha} &= -\Lambda_{\alpha} s_\alpha + \Delta \tau_{\alpha} + \bar{M}_{\alpha\alpha}(\ddot{\alpha}_d- \Phi_{\alpha} {\dot{e}_\alpha}),  \label{tau_alpha}\\
\Delta \tau_{\alpha} &= \begin{cases}
    \rho_{\alpha} \frac{s_{\alpha}}{||s_{\alpha}||}       & ~ \text{if } || s_{\alpha}|| \geq \varpi_\alpha\\
    \label{del_alpha}
    \rho_{\alpha} \frac{s_{\alpha}}{\varpi_\alpha}       & ~ \text{if } || s_{\alpha}|| < \varpi_\alpha\\
    \end{cases}
\end{align}
\end{subequations}
where $\Lambda_{\alpha}$ is a user-defined positive definite gain matrix and $\varpi_{\alpha} > 0$ is a scalar used to avoid chattering; $\rho_{\alpha}$ tackles system uncertainties, whose design will be discussed later.

 
Using Property 2 and the relation (\ref{split_2}) we have $||C_{\alpha}|| \leq ||C|| \leq \bar{c}||\dot{\chi}||$, $||g_{\alpha}|| \leq ||g|| \leq \bar{g}$ and $||d_{\alpha}|| \leq ||d|| \leq \bar{d}$. Using these relations, the inequalities $||\ddot{\chi}|| \geq ||\ddot{p}||$, $||\ddot{\chi}|| \geq ||\ddot{q}||$ and $||\ddot{\chi}|| \geq ||\ddot{\alpha}||$, $||\xi|| \geq ||\dot{e}||$, $||\xi|| \geq ||{e}||$ and substituting $\dot{\chi} = \dot{e} + \dot{\chi}_d$ into (\ref{calpha}) yields
\begin{align} 
||E_{\alpha}|| &\leq K_{\alpha0}^*  +K_{\alpha1}^*||\xi||+ K_{\alpha2}^*||\xi||^2 + K_{\alpha3}^*||\ddot{\chi}||\label{up_bound_alpha} 
%&\triangleq \mathbf{Y}^T \mathbf{K_{p}^*}
\end{align}

\begin{align*}
 K_{\alpha0}^* &= \bar{g} + \bar{d} + \bar{c}||\dot{\chi}_d||^2,\\
K_{\alpha1}^* &= 2\bar{c}|| \dot{\chi}_d||,~
K_{\alpha2}^* = \bar{c}, \\
K_{\alpha3}^* &=  (||(M_{\alpha\alpha} - \bar{M}_{\alpha\alpha} )||+ ||M^T_{p\alpha}|| + ||M^T_{q\alpha}||)
\end{align*}
are unknown scalars. Based on the upper bound structure in (\ref{up_bound_alpha}), the gain $\rho_{\alpha}$ in (\ref{del_alpha}) is designed as
\begin{align}
\rho_{\alpha} = \hat{K}_{\alpha0} + \hat{K}_{\alpha1}||\xi|| + \hat{K}_{\alpha2}||\xi||^2 + \hat{K}_{\alpha3}||\ddot{\chi}|| + \zeta_{\alpha} + \gamma_{\alpha} \label{rho_alpha} 
\end{align}
%\nonumber \\
%&\triangleq \mathbf{Y^T \hat{K}_{p}} + \zeta_p 

where $\hat{K}_{\alpha i}$ are the estimates of $K_{\alpha i}^*$ $i=0,1,2,3$, and $\zeta_{\alpha}$, $\gamma_{\alpha}$ are auxiliary gain used for closed-loop stabilization (cf. Remark \ref{rem_zeta}). The gains $\hat{K}_{\alpha i}$ are adapted via the following laws:
\begin{subequations}\label{adaptive_law_alpha}
\begin{align}
&\dot{\hat{K}}_{\alpha ia} = ||s_{\alpha}||||\xi||^i - \nu_{\alpha ia} \hat{K}_{\alpha ia}, \dot{\hat{K}}_{\alpha i\bar{a}} = 0,  i = 0,1,2
 \\
&\dot{\hat{K}}_{\alpha3a} =  ||s_{\alpha}||||\ddot{\chi}|| - \nu_{\alpha3a} \hat{K}_{\alpha3a}, \dot{\hat{K}}_{\alpha3\bar{a}} = 0\\
&\dot{\zeta}_{\alpha a} = -(1 + \hat{K}_{\alpha3a}||s_{\alpha}||||\ddot{\chi}||)\zeta_{\alpha a} + \bar{\epsilon}_{\alpha a}, \dot{\zeta}_{\alpha \bar{a}} = 0    \\ \label{zeta1}
& \dot{\gamma}_{\alpha a} = 0,  \dot{\gamma}_{\alpha\bar{a}} = \left(1+ \frac{\varrho_{\alpha\bar{a}}}{2}\sum \limits_{i=0}^{3} {\hat{K}_{\alpha i\bar{a}}}^2 \right) \gamma_{\alpha \bar{a}} +\epsilon_{\alpha \bar{a}} \\
& \text{with}~ \nu_{\alpha i} > \varrho_{\alpha}/2, i = 0,1,2\\
&~\hat{K}_{\alpha i} (t_0) > 0, ~\zeta_{\alpha } (t_0)  = \bar{\zeta}_{\alpha } > \bar{\epsilon}_{\alpha }, \gamma_{\alpha } (t_0) = \bar{\gamma}_{\alpha } > \epsilon_{\alpha }
\end{align}
\end{subequations}
where a and $\bar{a} \in \Omega \backslash {a}$ denote the active and inactive subsystems respectively; $\nu_{\alpha ia},\bar{\epsilon}_{\alpha a}, \epsilon_{\alpha\bar{a}}  \in\mathbb{R}^{+}$ are user-defined scalars and $\varrho_{\alpha} \triangleq \frac{\min \lbrace \lambda_{\min}(  \Lambda_{{\alpha}} ), \lambda_{\min}( \Phi_{{\alpha}} ) \rbrace}{\max \lbrace \bar{M}_{\alpha\alpha} 1 \rbrace}$ $\forall  \in \Omega$ and $t_0$ is the initial time and from \ref{} and intial condition \ref{}, it can be verified that $\exists \underline{\zeta}_{\alpha }, \underline{\gamma}_{\alpha } \in \mathbb{R}^{+}$ such that
\begin{subequations}
\begin{align}
&\hat{K}_{\alpha i}(t) \geq 0, ~~ 0 < \underline{\zeta}_{\alpha } \leq \zeta_{\alpha } (t) <  \bar{\zeta}_{\alpha }, \nonumber \\
\text{and}~~ & 0 < \underline{\gamma}_{\alpha } \leq \gamma_{\alpha } (t) <  \bar{\gamma}_{\alpha } ~~ \forall t \geq t_0.
\end{align}
\end{subequations}



\subsubsection{Desired Dynamics selection}

As mentioned earlier the most popular choice of dynamics to be imposed is the standard second order dynamics similar to a Mass-Spring-Damper system.

\begin{equation} \label{Impedance_dynamics}
\bar{M}_{\alpha\alpha}( \ddot{\alpha}(t) - \ddot{\alpha}_d(t)  ) + B_d ( \dot{\alpha}(t) - \dot{\alpha}_d(t)) + K_d ( {\alpha}(t) - {\alpha_d(t))} = {\tau}_{{\alpha}ext} - {\tau}_{{\alpha}d}  
\end{equation}

Where $\bar{M}_{\alpha\alpha}$ represents the desired Inertia matrix to be imposed; $B_d$ represents the desired damping; $K_d$ represents the desired spring constant for the system; ${r}_d$ represents the desired trajectory;${\tau}_{{\alpha}d} = J(\alpha)^{T} F_d$ where $J(\alpha)$ is the analytic jacobian matrix and ${F}_d$ represents the desired contact force.

Let ${\alpha}(t) - {\alpha}_d(t) \triangleq  {e_\alpha}(t)$ and ${\tau}_{{\alpha}ext} - {\tau}_{{\alpha}d}  \triangleq  {e_{\tau}}(t)$, then equation (\ref{Impedance_dynamics}) can be rewritten as 

\begin{equation}
    \bar{M}_{\alpha\alpha} \ddot{e_\alpha}(t) + B_d \dot{e_\alpha}(t) + K_d {e_\alpha}(t) = e_{\tau}
\end{equation}

and subsequently we can write it as

\begin{equation} \label{Simplified_Impedance_Dynamics}
 \ddot{e_\alpha}(t)  + {K_1}  \dot{e_\alpha}(t) +  {K_2} {e_\alpha}(t) = {K_3} e_{\tau}   
\end{equation}

where $K_1 = {\bar{M}_{\alpha\alpha}}^{-1} B_d$; $K_2 = {\bar{M}_{\alpha\alpha}}^{-1} K_d$; and $K_3 = {\bar{M}_{\alpha\alpha}}^{-1}$. 

\subsubsection{Control Problem Formulation}
Now as the intended dynamics are defined as \ref{Simplified_Impedance_Dynamics}, we define the deviation from the intended Impedance dynamics as $\Delta I$ (\ref{deviation_error})

\begin{equation} \label{deviation_error}
   \Delta I  \triangleq \ddot{e_\alpha}(t)  + {K_1}  \dot{e_\alpha}(t) +  {K_2} {e_\alpha}(t) - {K_3} e_{\tau}
\end{equation}
Our control objective is to reduce the Deviation Error $\Delta I$ to zero which will in turn enforce the dynamics (\ref{Simplified_Impedance_Dynamics}) and subsequently (\ref{Impedance_dynamics}) on the system.
\subsubsection{Sliding surface selection}

Considering the force and position tracking problem we define a sliding variable $\mathbf{s}$ as
\begin{equation} \label{s}
    s \triangleq \dot{e_\alpha}(t) + \lambda {e_\alpha} - {g_f}(t)
\end{equation}

where $g_f$ is an unknown function.

Then $\dot s$ is given as
\begin{equation} \label{s_dot}
    \dot s = \ddot{e_\alpha}(t) + \lambda {\dot e_\alpha} - \dot {g_f}(t)
\end{equation}

now, from (\ref{s_dot}) and (\ref{deviation_error})

% \begin{subequations}
% \begin{align}
%     \Delta I &= \dot s - \lambda \dot{e_\alpha} + \dot{g_f}(t) + K_1 \dot{e_\alpha}(t) + K_2 e_\alpha(t) - K_3 e_{\tau} \label{eq:sub1} \\
%     \Delta I &= \dot s - \lambda \dot{e_\alpha} + \dot{g_f}(t) + K_1 \dot{e_\alpha}(t) + K_2 e_\alpha(t) - K_3 e_{\tau} \label{eq:sub2}
% \end{align}
% \end{subequations}

\begin{equation} \label{devsol1}
    \Delta I = \dot s - \lambda \dot{e_\alpha} + \dot{g_f}(t) + K_1 \dot{e_\alpha}(t) + K_2 e_\alpha(t) - K_3 e_{\tau}
\end{equation}

\begin{equation}\label{devsol2}
    \Delta I = \dot s + (K_1 - \lambda) [\dot{e_\alpha} + \frac{K_2}{(K_1 - \lambda)} e_\alpha(t)] 
    + \dot{g_f}(t) - K_3 e_{\tau}
\end{equation}

\begin{equation} \label{devsol3}
    \Delta I = \dot s + \alpha [\dot{e_\alpha} + \beta e_\alpha(t)] 
    + \dot{g_f}(t) - K_3 e_{\tau}
\end{equation}


where $\alpha \triangleq K_1 - \lambda$ and $\beta \triangleq \frac{K_2}{(K_1 - \lambda)}$

Let $\beta = \lambda$, then $\lambda$ can be solved in terms of $K_1$ and $K_2$ using the following equation  
\begin{equation} \label{devsol4}
    \lambda(K_1 - \lambda) = K_2
\end{equation}
So, equation (\ref{devsol3}) becomes

\begin{equation} \label{devsol5}
    \Delta I = \dot s + \alpha [\dot{e_\alpha} + \lambda e_\alpha(t)] 
    + \dot{g_f}(t) - K_3 e_{\tau}
\end{equation}

using (\ref{devsol5}) and (\ref{s})

\begin{equation} 
    \Delta I = \dot s + \alpha [s + g_f(t)] 
    + \dot{g_f}(t) - K_3 e_{\tau}
\end{equation}
rearranging the terms,
\begin{equation} \label{devsol6}
    \Delta I = \dot s + \alpha s +\dot{g_f}(t) + \alpha g_f(t) - K_3 e_{\tau}
\end{equation}

let $g_f$ satisfy the equation
\begin{equation} \label{gf_dyn}
    \dot{g_f}(t) + \alpha g_f(t) - K_3 e_{\tau} = 0
\end{equation}
taking the Laplace transform we can see that

\begin{equation}
    G_f(\boldsymbol{S}) = \frac{K_3 e_{\tau}(\boldsymbol{S})}{\boldsymbol{S} + \alpha}
\end{equation}
where $\boldsymbol{S}$ is the Laplace Variable.
This is the standard structure of a Low-Pass filter, and we can define $g_f(t)$ to be a Low-Pass filtered force error signal and consider it as known if $e_{\tau}$ is known.

Hence, the sliding variable $s$ is now fully known.

Now, from (\ref{devsol6}) and (\ref{devsol7}) 
\begin{equation} \label{deviation_error2}
    \Delta I = \dot s + \alpha s
\end{equation}

\textit{Sub-Control Problem:} Now, if we can force $s$ and $\dot s$ to go to zero using any control strategy then we can ensure that the deviation error $\Delta I$  goes to zero, enforcing the desired impedance dynamics (\ref{Simplified_Impedance_Dynamics}) on the system.

\subsubsection{Controller Derivation and Adaptation}

For control design purpose, the UAM manipulator sub-dynamics (\ref{dynamics}c) is rearranged as
\begin{equation} \label{dynamics_alpha}
\bar{M}_{\alpha\alpha}\ddot{\alpha} + E_{\alpha} = \tau_{\alpha}   + \tau_{\alpha{ext}}    
\end{equation}
where $\bar{M}_{\alpha\alpha}$ is a user-defined constant positive definite matrix and  
\begin{equation} \label{Ealpha}
    E_\alpha \triangleq (M_{\alpha\alpha} - \bar{M}_{\alpha\alpha})\ddot{\alpha} +  M_{p\alpha}^T\ddot{p} + M_{q\alpha}^T\ddot{q} + C_\alpha\dot{\chi} + g_\alpha + d_\alpha 
\end{equation}
 

The selection of $\bar{M}_{\alpha \alpha}$ would be discussed later (cf. Remark \ref{mass}).

% \begin{itemize}
%     \item $M(q)$ is a positive definite matrix such that  $0 < \underline{m} I< M(q) < \overline{m} I$
%     \item  $M(q)$ and $C(q, \dot q)$ follow the property that $\dot M(t) - 2 C(t)$ is a skew-symmetric matrix. 
%     \item g(
% \end{itemize}

Now, from(\ref{s_dot}) 

\begin{equation}\label{controller_der_eq1}
    \dot{s}_\alpha = \ddot{\alpha}(t) - \ddot{\alpha_d}(t) + \lambda {\dot e_\alpha} - \dot {g_f}(t)
\end{equation}
Multiplying with $\bar{M}_{\alpha \alpha}$ on both we get

\begin{equation}\label{controller_der_eq2}
    \bar{M}_{\alpha\alpha} \dot s_\alpha = {\bar{M}}_{\alpha\alpha} \ddot{\alpha}(t) - \bar{M}_{\alpha\alpha} [\ddot{\alpha}_d - \lambda {\dot {e_\alpha}} + \dot {g_f}]
\end{equation}

Then from equation (\ref{dynamics_alpha}) we know,

\begin{equation} \label{}
\bar{M}_{\alpha\alpha}\ddot{{\alpha}} = - E_{\alpha} + \tau_{\alpha}   + \tau_{\alpha{ext}}       
\end{equation}

Combining this with (\ref{controller_der_eq2})

\begin{equation}
% \label{controller_der_eq2}
    \bar{M}_{\alpha\alpha} \dot s_\alpha = - E_{\alpha} + \tau_{\alpha}   + \tau_{\alpha{ext}}  - \bar{M}_{\alpha \alpha} [\ddot{\alpha_d} - \lambda {\dot e_\alpha} + \dot {g_f}]
\end{equation} 
Substituting $\dot{g_f}$ from (\ref{gf_dyn}) into (\ref{controller_der_eq2}) we get

\begin{equation}
% \label{controller_der_eq2}
    \bar{M}_{\alpha\alpha} \dot s_\alpha = - E_{\alpha} + \tau_{\alpha}   + \tau_{\alpha{ext}}  - \bar{M}_{\alpha \alpha} [\ddot{\alpha_d} - \lambda {\dot e_\alpha} - \alpha g_f(t) + K_3 e_{\tau}]
\end{equation} 


The control law is proposed as
\begin{subequations}\label{ct3}
\begin{align}
\tau_{\alpha} &= -\Lambda_{\alpha} s_\alpha + \Delta \tau_{\alpha} - \tau_{\alpha{ext}} + \bar{M}_{\alpha \alpha} [\ddot{\alpha_d} - \lambda {\dot e_\alpha} - \alpha g_f(t) + K_3 e_{\tau}],  \label{tau_alpha}\\
\Delta \tau_{\alpha} &= \begin{cases}
    \rho_{\alpha} \frac{s_{\alpha}}{||s_{\alpha}||}       & ~ \text{if } || s_{\alpha}|| \geq \varpi_\alpha\\
    \label{del_alpha}
    \rho_{\alpha} \frac{s_{\alpha}}{\varpi_\alpha}       & ~ \text{if } || s_{\alpha}|| < \varpi_\alpha\\
    \end{cases}
\end{align}
\end{subequations}
where $\Lambda_{\alpha}$ is a user-defined positive definite gain matrix and $\varpi_{\alpha} > 0$ is a scalar used to avoid chattering; $\rho_{\alpha}$ tackles system uncertainties, whose design will be discussed later.

 
Using Property 2 and the relation (\ref{split_2}) we have $||C_{\alpha}|| \leq ||C|| \leq \bar{c}||\dot{\chi}||$, $||g_{\alpha}|| \leq ||g|| \leq \bar{g}$ and $||d_{\alpha}|| \leq ||d|| \leq \bar{d}$. Using these relations, the inequalities $||\ddot{\chi}|| \geq ||\ddot{p}||$, $||\ddot{\chi}|| \geq ||\ddot{q}||$ and $||\ddot{\chi}|| \geq ||\ddot{\alpha}||$, $||\xi|| \geq ||\dot{e}||$, $||\xi|| \geq ||{e}||$ and substituting $\dot{\chi} = \dot{e} + \dot{\chi}_d$ into (\ref{Ealpha}) yields the upper bound structure for $E_\alpha$ as
\begin{align} 
||E_{\alpha}|| &\leq K_{\alpha0}^*  +K_{\alpha1}^*||\xi||+ K_{\alpha2}^*||\xi||^2 + K_{\alpha3}^*||\ddot{\chi}||\label{up_bound_alpha} 
%&\triangleq \mathbf{Y}^T \mathbf{K_{p}^*}
\end{align}

\begin{align*}
 K_{\alpha0}^* &= \bar{g} + \bar{d} + \bar{c}||\dot{\chi}_d||^2,\\
K_{\alpha1}^* &= 2\bar{c}|| \dot{\chi}_d||,~
K_{\alpha2}^* = \bar{c}, \\
K_{\alpha3}^* &=  (||(M_{\alpha\alpha} - \bar{M}_{\alpha\alpha} )||+ ||M^T_{p\alpha}|| + ||M^T_{q\alpha}||)
\end{align*}
are unknown scalars. 
Based on the upper bound structure in (\ref{up_bound_alpha}), the gain $\rho_{\alpha}$ in (\ref{del_alpha}) is designed as
\begin{align}
\rho_{\alpha} = \hat{K}_{\alpha0} + \hat{K}_{\alpha1}||\xi|| + \hat{K}_{\alpha2}||\xi||^2 + \hat{K}_{\alpha3}||\ddot{\chi}|| + \zeta_{\alpha} + \gamma_{\alpha} \label{rho_alpha} 
\end{align}
%\nonumber \\
%&\triangleq \mathbf{Y^T \hat{K}_{p}} + \zeta_p 

where $\hat{K}_{\alpha i}$ are the estimates of $K_{\alpha i}^*$ $i=0,1,2,3$, and $\zeta_{\alpha}$, $\gamma_{\alpha}$ are auxiliary gain used for closed-loop stabilization (cf. Remark \ref{rem_zeta}). The gains $\hat{K}_{\alpha i}$ are adapted via the following laws:
\begin{subequations}\label{adaptive_law_alpha}
\begin{align}
&\dot{\hat{K}}_{\alpha ia} = ||s_{\alpha}||||\xi||^i - \nu_{\alpha ia} \hat{K}_{\alpha ia}, \dot{\hat{K}}_{\alpha i\bar{a}} = 0,  i = 0,1,2
 \\
&\dot{\hat{K}}_{\alpha3a} =  ||s_{\alpha}||||\ddot{\chi}|| - \nu_{\alpha3a} \hat{K}_{\alpha3a}, \dot{\hat{K}}_{\alpha3\bar{a}} = 0\\
&\dot{\zeta}_{\alpha a} = -(1 + \hat{K}_{\alpha3a}||s_{\alpha}||||\ddot{\chi}||)\zeta_{\alpha a} + \bar{\epsilon}_{\alpha a}, \dot{\zeta}_{\alpha \bar{a}} = 0    \\ \label{zeta1}
& \dot{\gamma}_{\alpha a} = 0,  \dot{\gamma}_{\alpha\bar{a}} = \left(1+ \frac{\varrho_{\alpha\bar{a}}}{2}\sum \limits_{i=0}^{3} {\hat{K}_{\alpha i\bar{a}}}^2 \right) \gamma_{\alpha \bar{a}} +\epsilon_{\alpha \bar{a}} \\
& \text{with}~ \nu_{\alpha i} > \varrho_{\alpha}/2, i = 0,1,2\\
&~\hat{K}_{\alpha i} (t_0) > 0, ~\zeta_{\alpha } (t_0)  = \bar{\zeta}_{\alpha } > \bar{\epsilon}_{\alpha }, \gamma_{\alpha } (t_0) = \bar{\gamma}_{\alpha } > \epsilon_{\alpha }
\end{align}
\end{subequations}
where a and $\bar{a} \in \Omega \backslash {a}$ denote the active and inactive subsystems respectively; $\nu_{\alpha ia},\bar{\epsilon}_{\alpha a}, \epsilon_{\alpha\bar{a}}  \in\mathbb{R}^{+}$ are user-defined scalars and $\varrho_{\alpha} \triangleq \frac{\min \lbrace \lambda_{\min}(  \Lambda_{{\alpha}} ), \lambda_{\min}( \Phi_{{\alpha}} ) \rbrace}{\max \lbrace \bar{M}_{\alpha\alpha} 1 \rbrace}$ $\forall  \in \Omega$ and $t_0$ is the initial time and from \ref{} and intial condition \ref{}, it can be verified that $\exists \underline{\zeta}_{\alpha }, \underline{\gamma}_{\alpha } \in \mathbb{R}^{+}$ such that
\begin{subequations}
\begin{align}
&\hat{K}_{\alpha i}(t) \geq 0, ~~ 0 < \underline{\zeta}_{\alpha } \leq \zeta_{\alpha } (t) <  \bar{\zeta}_{\alpha }, \nonumber \\
\text{and}~~ & 0 < \underline{\gamma}_{\alpha } \leq \gamma_{\alpha } (t) <  \bar{\gamma}_{\alpha } ~~ \forall t \geq t_0.
\end{align}
\end{subequations}


% \bibliographystyle{IEEEtran}
% \bibliography{root}


\end{document}

\nomenclature{\(b_r\)}{Width of right side module}

